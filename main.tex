\documentclass{article}

\title{A Beginner's Guide to Writing Documentation}
\author{@0ri0nRo}
\date{\today}

\begin{document}

\maketitle

\section{Introduction}
Writing documentation is essential for any project. It can be daunting, but this guide aims to simplify the process.

\section{Why Write Docs}
\begin{itemize}
    \item Code becomes hard to understand after some time.
    \item Documentation helps users understand the purpose of the code.
    \item Good documentation encourages people to use and contribute to your project.
\end{itemize}

\section{Benefits of Documentation}
\begin{itemize}
    \item Facilitates understanding and future maintenance of the code.
    \item Builds confidence in releasing code into the open source community.
    \item Increases the likelihood of people using and contributing to the project.
    \item Improves the design of the code and enhances the developer's writing skills.
\end{itemize}

\section{Getting Started}
\begin{itemize}
    \item Start simple and gradually expand documentation as needed.
    \item Use version-controlled plain text for documentation.
    \item Begin with a basic README file and expand from there.
\end{itemize}

\section{What to Include}
\begin{itemize}
    \item Explanation of the project's purpose and problem it solves.
    \item Code examples, links to code repository, and issue tracker.
    \item FAQs, support channels, and information for contributors.
    \item Installation instructions and project's license details.
\end{itemize}

\section{Contributes}

If you want to contribute to the project, follow these steps:
\begin{itemize}
    \item Fork the repository
    \item Create a new branch (`git checkout -b feature`)
    \item Commit your changes (`git commit -am 'Add new feature'`)
    \item Push the branch (`git push origin feature`)
    \item Submit a pull request
\end{itemize}

\section{Project name}
The project aims to provide clear documentation guidelines for beginners, offering insights into why documentation is essential, how it benefits both users and developers, and practical tips for writing effective documentation.

\section{Key Points}

\begin{itemize}
    \item Purpose of Documentation: Documentation serves as a roadmap for understanding code, transferring the 'why' behind code decisions, and enabling future contributions.
    \item Benefits of Documentation: Good documentation increases code usability, encourages contributions, improves code design, and enhances technical writing skills.
    \item Getting Started: Beginners are encouraged to start with simple documentation using plain text tools, such as Markdown or reStructuredText, and to maintain a version-controlled workflow.
    \item Content Guidelines: Documentation should address the project's purpose, provide usage examples, offer links to code repositories and issue trackers, include FAQs, support information, contribution guidelines, installation instructions, and details about the project's license.
    \item Contributing to the Project: Contributors are guided through forking the repository, creating a new branch, committing changes, pushing the branch, and submitting a pull request.
    By following these guidelines, beginners can create comprehensive documentation for their projects, ensuring clarity and accessibility for users and contributors alike.
\end{itemize}


\section{Conclusion}
Writing documentation is crucial for the success of any project. Start with the basics, keep it updated, and embrace imperfection.

\end{document}
